\newcounter{nuserstory}
\newcounter{nusecase}

\newcommand{\userstory}[4]{%
    \refstepcounter{nuserstory}
    \subsection{#1}
    \label{userstory:\thenuserstory}
    \hangindent=40pt
    \textbf{\textit{As a}} #2,\\
    \textbf{\textit{I want to}} #3,\\
    \textbf{\textit{so that}} #4.
}
\newenvironment{usecase}[1]
{
    \refstepcounter{nusecase}%
    \subsection{Use Case \thenusecase: #1}%
    \label{usecase:\thenusecase}%
}{}

% With user stories (\userstory) you can reference them back later with \ref{userstory:n}
% For example, first user story in this file can be referred with \ref{userstory:1}

\chapter{Requirement Analysis}
\label{chap:requirement-analysis}

\section{Stakeholder Analysis}
\label{section:stakeholder-analysis}

\subsection{Primary Stakeholders}
\label{subsection:primary-stakeholders}

\begin{enumerate}[leftmargin=80pt]
    \item \textbf{Users:} The primary stakeholders are the users of our multi-camera tracking system, including security personnel, facility managers, emergency response coordinators, and research analysts.  Their interaction with the system is crucial for effective monitoring, incident response, and data-driven decision-making.  For a detailed specification of our users, see \ref{section:target-user}.
    \item \textbf{System Administrators:}  These individuals are responsible for the installation, configuration, and maintenance of the system.  Their needs include ease of deployment, system stability, and access to diagnostic tools.
    \item \textbf{University/Factory Management:}  These stakeholders are interested in the overall benefits of the system, such as improved safety, optimized resource allocation, and enhanced operational efficiency.  They require reports and data summaries to justify the investment and track performance.
\end{enumerate}

\section{User Stories}
\label{section:user-stories}

%% User Stories Start

\userstory{Track Individuals Across Cameras%
}{security officer%
}{seamlessly track individuals as they move between different camera views%
}{I can maintain continuous surveillance and respond quickly to incidents}

\userstory{Identify Individuals After Re-Appearance%
}{security officer%
}{re-identify individuals who have temporarily left the camera network's view and reappeared%
}{I can maintain a consistent track of individuals even with gaps in coverage}

\userstory{Visualize Movement Patterns%
}{facility operations manager%
}{view a map displaying the movement trajectories of individuals across the monitored area%
}{I can understand how spaces are being used and identify potential bottlenecks or areas of congestion}

\userstory{Review Historical Tracking Data%
}{research analyst%
}{access and analyze historical tracking data for individuals and groups%
}{I can understand movement patterns, space utilization, and other behavioral insights}

\userstory{Export Tracking Data%
}{System Administrator%
}{be able to export the tracking data%
}{I can integrate to the existing system in the organization}

\userstory{Select a Specific Individual for Tracking%
}{security officer%
}{select a specific individual from a live camera feed or a list of detected persons for prioritized tracking and detailed information display%
}{I can focus attention and receive more detailed information on a person of interest, while others are still tracked in the background}

\userstory{Deselect a Tracked Individual%
}{security officer%
}{remove the prioritized tracking and detailed display from a currently selected individual%
}{I can shift my focus to other individuals or situations, returning the deselected individual to the general tracking pool}

\userstory{View Multiple Camera Feeds Simultaneously%
}{Security Officer%
}{to view and monitor the live camera feeds%
}{I could see the environment in real time}

%% User Stories End

\section{Use Cases}
\label{section:use-cases}
%    \begin{usecase}{Track a Suspicious Individual}
%        \textbf{Actors:} Security Officer
%        \textbf{Description:} A security officer observes suspicious behavior from an individual on one camera.  The officer uses the system to initiate tracking of that individual, maintaining their identity as they move across multiple camera views.  The system provides real-time location updates and alerts the officer if the individual enters a restricted area.
%        \textbf{Pre-conditions:} The system is operational and cameras are calibrated. The security officer is logged in.
%        \textbf{Post-conditions:} The individual's movements are tracked and recorded.  Alerts are generated based on predefined rules.
%    \end{usecase}
%
%     \begin{usecase}{Respond to an Emergency}
%        \textbf{Actors:} Emergency Response Coordinator
%        \textbf{Description:}  During an emergency (e.g., fire alarm), the coordinator uses the system to monitor the evacuation of a building.  The system displays the locations of individuals, identifies areas of high density, and helps the coordinator direct resources to assist those who may need help. The coordinator can search a person by their appearence to effectively pin point people in need.
%         \textbf{Pre-conditions:} The system is operational. Emergency protocols are defined within the system.
%        \textbf{Post-conditions:} Evacuation is monitored, and resources are deployed effectively.  Data is recorded for post-incident analysis.
%    \end{usecase}
%
%    \begin{usecase}{Analyze Space Utilization}
%        \textbf{Actors:} Facility Operations Manager
%        \textbf{Description:} A facility manager uses the system to analyze historical movement data within a building. The system generates heatmaps showing areas of high and low traffic, identifies common pathways, and provides data on occupancy density over time.  This information is used to optimize space allocation, improve traffic flow, and inform decisions about renovations or resource placement.
%        \textbf{Pre-conditions:} The system has been collecting data for a sufficient period.
%        \textbf{Post-conditions:}  Reports and visualizations are generated, providing insights into space utilization.
%    \end{usecase}

% Use Stories End

\section{Use Case Diagram}
\label{section:use-case-diagram}

% \begin{figure}[h!]
%     \centering
%     \begin{tikzpicture}

%         \begin{umlsystem}[x=4]{KU Eater}
%             \umlusecase[name=signup, y=4.5]{Sign Up}
%             \umlusecase[name=login, y=3]{Login}
%             \umlusecase[name=viewrec, y=1, width=2.5cm]{View Recommendations}
%             \umlusecase[name=genrec, x=6.5, y=2, width=2.5cm]{Generate Recommendations}
%             \umlusecase[name=menuinfo, x=4, y=0.25]{See Menu Info}
%             \umlusecase[name=searchmenu, x=1, y=-1]{Search Menu Items}
%             \umlusecase[name=userrec, x=6.5, y=-4, width=3.5cm]{Personalized Recommendations}
%             \umlusecase[name=trendrec, x=6.5, y=4.5, width=2.5cm]{Trending Recommendations}
%             \umlusecase[name=readreview, y=-2.5]{Read Reviews}
%             \umlusecase[name=writereview, x=-0.25, y=-4]{Write Review}
%             \umlusecase[name=rateitem, x=1, y=-5.5]{Rate Menu Items}
%             \umlusecase[name=editprofile, x=3, y=-7]{Edit Profile}
%         \end{umlsystem}

%         \umlactor[y=-1]{User}
%         \umlactor[y=3]{Guest}
%         \umlactor[y=-5]{Member}
%         \umlactor[x=15]{System}

%         \umlassoc{Guest}{login}
%         \umlassoc{Guest}{signup}
%         \umlassoc{User}{viewrec}
%         \umlassoc{User}{menuinfo}
%         \umlassoc{User}{searchmenu}
%         \umlassoc{User}{readreview}
%         \umlassoc{Member}{writereview}
%         \umlassoc{Member}{rateitem}
%         \umlassoc{Member}{editprofile}
%         \umlassoc{System}{genrec}
%         \umlinclude{viewrec}{genrec}
%         \umlinclude{userrec}{writereview}
%         \umlinclude{userrec}{rateitem}
%         \umlinclude{userrec}{editprofile}
%         \umlinherit{Guest}{User}
%         \umlinherit{Member}{User}
%         \umlinherit{userrec}{genrec}
%         \umlinherit{trendrec}{genrec}

%     \end{tikzpicture}
%     \caption{Use Case Diagram of KU Eater}
%     \label{fig:use-case-diagram}
%     \vspace*{-1in}
% \end{figure}

% Use case diagram (Figure \ref{fig:use-case-diagram}) is a visualization of functionalities that KU Eater provides for end-users. We decided to break
% down the users \textit{(on the left side of figure)} into three groups:

% \begin{itemize}
%     \item \textbf{\textit{User}}---a default group of users which has all the basic functionalities including: Viewing Recommendations, Seeing Menu Details,
%     Searching for Menu Items, and Reading Reviews.
%     \item \textbf{\textit{Guest}}---a group of user which extends from base functionalities, they are able to Sign Up and Login.
%     \item \textbf{\textit{Member}}---an authorized group of user, being able to Write Reviews, Rate Menu Items and Edit their Profiles.
% \end{itemize}

\section{Use Case Model}
\label{section:use-case-model}

% Use Cases Start

% \begin{usecase}{Viewing the Menu}
%     \textbf{Actors:} Tom (Cafeteria User), KU Eater (System)
    
%     \textbf{Description:} Tom wants to see the food items being offered at the cafeteria.

%     \textbf{Scenario:}

%     \begin{enumerate}[leftmargin=80pt]
%         \item Tom opens KU Eater application on his device.
%         \item System renders the main page which contains recommendations and browse-able list of menus and stalls.
%         \item Tom can now browse the menu of each stalls and choose his favorite food.
%     \end{enumerate}
% \end{usecase}

% \begin{usecase}{Seeing Recommendations on Trending Menu}
%     \textbf{Actors:} Tom (Cafeteria User), KU Eater (System)
    
%     \textbf{Description:} Tom doesn't know what to eat, so he wants to find any stalls that could be good. The metrics
%     for good food is popularity.

%     \textbf{Scenario:}

%     \begin{enumerate}[leftmargin=80pt]
%         \item Tom opens KU Eater application on his device.
%         \item System starts to generate lists of recommendations, one of the list "Now Trending" scores the recommendations
%         based on average user ratings and review scores.
%         \item System renders back to Tom with an amount of topmost trending stalls and their menus.
%         \item Tom clicks on "See More."
%         \item System renders a scrollable list of more trending stalls.
%         \item Tom can now browse a list of menu items which are trending.
%     \end{enumerate}
% \end{usecase}

% \begin{usecase}{Seeing Personalized Recommendations}
%     \textbf{Actors:} Amber (Cafeteria User), KU Eater (System)
    
%     \textbf{Description:} Amber has been using KU Eater for a while and is now a member, she's has a pattern of liking seafood.
%     She wants to discover food choices available in the app.

%     \textbf{Preconditions:} Amber is a member and has rated and loved seafood in the cafeteria.

%     \textbf{Scenario:}

%     \begin{enumerate}[leftmargin=80pt]
%         \item Amber opens KU Eater application on her device.
%         \item System starts to generate lists of recommendations, one of the list "For You" scores recommendations
%         based on similarity of user preferences and menu database.
%         \item System renders back to Amber with an amount of relevant stalls and their menus.
%         \item Amber clicks on "See More."
%         \item System renders a scrollable list of more suitable stalls.
%         \item Amber can now browse a list of menu items which are mostly seafood and tailored for her.
%     \end{enumerate}
% \end{usecase}

% \begin{usecase}{Searching for a Specific Dish}
%     \textbf{Actors:} John (Cafeteria User), KU Eater (System)
    
%     \textbf{Description:} John wants to have boat noodles for lunch. He's using KU Eater to speed up
%     his process of finding.

%     \textbf{Scenario:}

%     \begin{enumerate}[leftmargin=80pt]
%         \item John opens KU Eater application on his device.
%         \item System renders the main page with search bar at the top.
%         \item John clicks on Search bar and types in "boat noodles."
%         \item System looks for stalls that sells boat noodles and renders a collective list of relevant stalls.
%         \item John will now see many stalls selling boat noodles.
%     \end{enumerate}
% \end{usecase}

% \begin{usecase}{Filtering with Dietary Restrictions}
%     \textbf{Actors:} Harry (Cafeteria User), KU Eater (System)

%     \textbf{Description:} Harry is allergic to bell peppers, however he wants to have stir-fry which some stalls cook with bell peppers.
%     He wants to find stalls that serve his way.

%     \textbf{Scenario:}

%     \begin{enumerate}[leftmargin=80pt]
%         \item Harry opens KU Eater application on his device.
%         \item System renders the main page with search bar at the top.
%         \item Harry clicks on Search bar and types in "stir-fry."
%         \item Harry also puts in filter as he doesn't want to include "bell peppers."
%         \item System looks for stalls that sells stir-fry with no bell peppers and renders a collective list of relevant stalls.
%         \item Harry will be able to browse the many relevant stalls.
%     \end{enumerate}
% \end{usecase}

% \begin{usecase}{Reading Reviews on Specific Stall}
%     \textbf{Actors:} Tom (Cafeteria User), KU Eater (System)

%     \textbf{Description:} Tom founds a trending dish on three stalls, however he wants to compare between each stalls.

%     \textbf{Scenario:}

%     \begin{enumerate}[leftmargin=80pt]
%         \item Tom opens KU Eater application on his device.
%         \item System renders the main page which contains recommendations and browse-able list of menus and stalls.
%         \item Tom sees three stalls that sell fried chicken---which piques his interest.
%         \item Tom clicks on one of the stalls.
%         \item System retrieves the most relevant reviews which are for fried chicken and renders them as a list.
%         \item Tom can now read the reviews on a specific stall.
%     \end{enumerate}
% \end{usecase}

% \begin{usecase}{Writing Review for a Dish in Stall}
%     \textbf{Actors:} Mike (Cafeteria User), KU Eater (System)

%     \textbf{Description:} Mike has just finished his meal and would like to get his thoughts on the dish to the platform.

%     \textbf{Preconditions:} Mike is a member on KU Eater.

%     \textbf{Scenario:}

%     \begin{enumerate}[leftmargin=80pt]
%         \item Mike opens KU Eater application on his device.
%         \item System renders the main page.
%         \item Mike goes to the correct stall page and clicks on his eaten menu item.
%         \item System renders information about his meal, including ingredients and other reviews.
%         \item Mike clicks on "Add Review," and writes his thoughts before hitting submit.
%         \item System redirects Mike back to the same menu item page with his comment now added.
%     \end{enumerate}
% \end{usecase}

% \begin{usecase}{Rating a Dish in Stall}
%     \textbf{Actors:} Carol (Cafeteria User), KU Eater (System)

%     \textbf{Description:} Carol has just finished her meal and wanted to help the stall gain more traction, but is lazy to be writing reviews.

%     \textbf{Preconditions:} Carol is a member on KU Eater.

%     \textbf{Scenario:}

%     \begin{enumerate}[leftmargin=80pt]
%         \item Carol opens KU Eater application on her device.
%         \item System renders the main page.
%         \item Carol goes to the correct stall page and finds her menu item.
%         \item Carol clicks on "Thumbs Up" button.
%         \item System records her rating and uses it to re-rank the stalls.
%     \end{enumerate}
% \end{usecase}

% % Use Cases End

% \newpage

\section{User Interface Design}
\label{section:user-interface-design}

% The main concept is that KU Eater's interface design prioritizes user-friendliness and an intuitive navigation experience,
% taking inspiration from various successful application platforms. While KU Eater is a web-based application,
% the user interface is presented in a mobile frame to demonstrate the versatility and mobile-first approach that aligns with
% the on-the-go lifestyle of Kasetsart University students.\footnote{Application mockup demonstration is available at: \url{https://www.youtube.com/watch?v=4CBDDqBZ2SA}}
% The figures below are the designs that aligns with our philosophy,

% % Mockup Start

% \begin{figure}[h!]
%     \begin{minipage}{.5\textwidth}
%         \centering
%         \includegraphics[height=0.4\textheight,keepaspectratio]{kueater/mockup/9. Home1 Page.PNG}
%     \end{minipage}%
%     \begin{minipage}{.5\textwidth}
%         \centering
%         \includegraphics[height=0.4\textheight,keepaspectratio]{kueater/mockup/10. Home2 Page.PNG}
%     \end{minipage}
%     \caption{Mockup of Homepage}
%     \vspace*{-\baselineskip}
% \end{figure}

% \begin{figure}[h!]
%     \begin{minipage}{.5\textwidth}
%         \centering
%         \includegraphics[height=0.4\textheight,keepaspectratio]{kueater/mockup/7. Preference Starter2 Page.PNG}
%         \small{\caption{Mockup of User Onboarding}}
%     \end{minipage}%
%     \begin{minipage}{.5\textwidth}
%         \centering
%         \includegraphics[height=0.4\textheight,keepaspectratio]{kueater/mockup/11. Searching Page.PNG}
%         \small{\caption{Mockup of Search Function}}
%     \end{minipage}
%     \vspace*{-\baselineskip}
% \end{figure}

% \begin{figure}[h!]
%     \begin{minipage}{.5\textwidth}
%         \centering
%         \includegraphics[height=0.4\textheight,keepaspectratio]{kueater/mockup/13. Stall Profile1 Page.PNG}
%         \small{\caption{Mockup of Storefront}}
%     \end{minipage}%
%     \begin{minipage}{.5\textwidth}
%         \centering
%         \includegraphics[height=0.4\textheight,keepaspectratio]{kueater/mockup/14. Stall Profile1 Page.PNG}
%         \small{\caption{Mockup of Menu List}}
%     \end{minipage}
%     \vspace*{-\baselineskip}
% \end{figure}


% \begin{figure}[h!t]
%     \begin{minipage}{.5\textwidth}
%         \centering
%         \includegraphics[height=0.4\textheight,keepaspectratio]{kueater/mockup/15. Rating Page.PNG}
%         \small{\caption{Mockup of Review List}}
%     \end{minipage}%
%     \begin{minipage}{.5\textwidth}
%         \centering
%         \includegraphics[height=0.4\textheight,keepaspectratio]{kueater/mockup/16. Rating Modal.PNG}
%         \small{\caption{Mockup of Review Writing}}
%     \end{minipage}
% \end{figure}

% % Mockup End

\newpage

\section{Target and Development System}
\label{section:development-system}

\subsection{Data Preparation}
\label{subsection:data-preparation-system}
% In order to prepare data for KU Eater, proper tooling is needed both for data storage and retrieval. The following list is the technologies we used in data preparation:

% \begin{itemize}[leftmargin=80pt]
%     \item \textbf{Google Sheets:} In order to record interview results such as stall information, list of menu items, we decided to use Google Sheets as our storage for raw data\footnote{The spreadsheet containing the data: \url{https://docs.google.com/spreadsheets/d/1oQ_PykNVvtV5KKtKOrva9aiZeK4S42EuFP9S6qaO4cg}}.
%     The advantage of this tool is that we have faster access to data, especially when we are doing interviews with data holders.
%     \item \textbf{PostgreSQL:} PostgreSQL is a relational database system, one of the advantages that PostgreSQL offers is the synchronous replication, we can use that feature in order to separate between the master data store, and sandbox database for training ML models.
%     PostgreSQL is used as our main datastore. PostgreSQL also has vector support for AI-driven applications.
% \end{itemize}

% \subsection{Recommendation Engine}
% \label{subsection:recommendation-engine}
% KU Eater at its core features a recommendation system, in order to build a consistent and fast service, we chose the following technologies:

% \begin{itemize}[leftmargin=80pt]
%     \item \textbf{gRPC:} A high performant protocol to do Remote Procedure Calls (RPC) that allows us to share schemas between
%     services. This allows for consistent data throughout the application's scope and predictable behavior.
%     \item \textbf{Tonic:} A gRPC server based on the Rust programming language. It is performant and can be
%     used to serve data asynchronously for a near real-time experience.
%     \item \textbf{PostgreSQL:} Like in the step of Data Preparation (\ref{subsection:data-preparation-system}), we can use replication to our advantage to update data in real-time
%     both from autonomous systems or administrator edits.
% \end{itemize}


% \subsection{Web Application}
% \label{subsection:web-application}
% KU Eater as a web-based mobile application, needs an interface designed with a target of mobile phone users in mind.
% In order to complement our requirements, our approach to the development process of the application must be mobile-first.
% The following is the technology stack of KU Eater web application:

% \begin{itemize}[leftmargin=80pt]
%     \item \textbf{React Native:} React Native is a framework based on React that offers powerful data fetching functionalities, built-in caching using existing web technologies etc. \\
%     The framework is native to Javascript, and can ensure its compatibility across devices. It is used as our frontend for the application and a mediator between the backend.
% \end{itemize}